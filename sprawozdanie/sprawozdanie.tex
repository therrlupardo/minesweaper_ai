\documentclass[letterpaper,12pt]{article}
\usepackage{polski}
\usepackage[utf8]{inputenc}
\usepackage[T1]{fontenc}
\usepackage{tabularx} % extra features for tabular environment
\usepackage{amsmath}  % improve math presentation
\usepackage{mathtools}
\usepackage{graphicx} % takes care of graphic including machinery
\usepackage[margin=1in,letterpaper]{geometry} % decreases margins
\usepackage{cite} % takes care of citations
\usepackage[final]{hyperref} % adds hyper links inside the generated pdf file
\usepackage{float}
\usepackage{tabto}

\hypersetup{
	colorlinks=true,       % false: boxed links; true: colored links
	linkcolor=blue,        % color of internal links
	citecolor=blue,        % color of links to bibliography
	filecolor=magenta,     % color of file links
	urlcolor=blue         
}


\graphicspath{ {./images/} }

\begin{document}
\title{Rozwiązywanie Sapera z pomocą sztucznej inteligencji}
\date{04.05.2019}
\author{ Mateusz Buchajewicz, Dominika Zembrzuska, Kacper Ziółkowski}
\maketitle

\section{Wprowadzenie}
Celem projektu jest utworzenie trzech algorytmów rozwiązujących popularną grę Saper.\\
\hfill \break
\section{Zasady sapera}
Celem gry jest oznaczenie wszystkich min na planszy. Domyślne warianty gry obejmują
planszę małą (o wymiarach 10x10 i zawierającą 10 min), średnią (16x16, 40 min)
oraz dużą (16x30, 99 min). Gracz może oznaczać pola z miną prawym przyciskiem
myszy oraz lewym pola bez niej. Odkrycie pola bez miny odsłania je,
wyświetlając na nim jego wartość od 1 do 8, równoważną ilości
min na polach sąsiadujących. Ponadto zostają też odsłonięte wszystkie są
siadujące pola, które nie mają wartość 0 oraz te z nimi graniczące. Możliwe też jest 
naciśnięcie pola numerycznego prawym i lewym przyciskiem myszy jednocześnie, gdy oznaczone
zostały wszystkie miny w jego sąsiedztwie. Powoduje to odsłonięcie pozostałych nieodsłoniętych pół.
Gra kończy się wygraną, jeśli gracz oznacz poprawnie wszystkie miny i pola bez nich.
Gracz przegrywa, jeśli wskaże pole zawierające minę jako pole bez niej.\\
\hfill \break

\section{Wykorzystane technologie}
Projekt został napisany w języku Python, korzysta z wersji online gry Saper znajdującej się pod adresem $minesweeperonline.com$. Obsługa okna przeglądarki zrealizowana 
została z pomocą pakietu selenium. Sieć neuronowa została utworzona przy pomocy pakietu tensorflow, a do uzyskiwania rozwiązań przy zastosowaniu zasad napisanych w Prologu
użyto pakietu PyDataLog.
\newpage
\section{Opisy algorytmów}
\subsection{Bot generujący dane}

Bot działa w oparciu o 3 główne metody rozwiązujące:
\begin{itemize}
    \item{Rozwiązywanie za pomocą prostej logiki}
    \item{Rozwiązywanie za pomocą układów równań}
    \item{Strzelanie w losowe pole}    \\
\end{itemize}
\subsubsection*{Rozwiązywanie za pomocą prostej logiki}
Sprawdza po kolei wszystkie pola planszy, zawierające wartość numeryczną. 
Jeśli liczba pól sąsiadujących ze sprawdzanym polem, a nieoznaczonych jako
"bez miny" (czyli zarówno oznaczonych jako "miny", jak i pustych) jest taka
jak wartość pola, to wszystkie te pola są oznaczane jako miny.
Natomiast jeżeli liczba oznaczonych min na polach sąsiadujących jest równa wartości 
sprawdzanego pola, to wszystkie pozostałe pola sąsiadujące zostają oznaczone jako "bez miny"
\hfill \break
\subsubsection*{Rozwiązywanie za pomocą układów równań}
Metoda opiera się na ułożeniu układu równań z pól na planszy oraz rozwiązaniu go. 
Algorytm przekształca układ równań na macierz n x m, gdzie n jest liczbą oznaczonych 
pól z wyznaczoną wartością, a m jest liczbą różnych pól nieoznaczonych z nimi sąsiadujących.\\
\newpage
Dla przykładu, poniższa plansza:\\
\begin{figure}[h]
    \centering
    \includegraphics[scale=0.65]{matrix_first.png}
    \caption{Przykładowa plansza sapera}
\end{figure} \\
Zostanie przekształcona do postaci:\\
\begin{figure}[h]
    \centering
    \includegraphics[scale=0.5]{matrix_marked.png}
    \caption{Przykładowa plansza sapera z polami oznaczonymi jako zmienne}
\end{figure} \\

Otrzymany układ równań: \\
$$
\begin{cases}
 A + B = 1 \\
 A + B + C = 1 \\
 D + E = 1 \\
 E + F = 1 \\
 A + B + C + F = 1 \\
\end{cases}
$$
\newpage
Co z kolei można zapisać w postaci macierzy: \\
\[
\begin{bmatrix}
    1 & 1 & 0 & 0 & 0 & 0 \\
    1 & 1 & 1 & 0 & 0 & 0 \\
    0 & 0 & 0 & 1 & 1 & 0 \\
    0 & 0 & 0 & 0 & 1 & 1 \\
    1 & 1 & 1 & 0 & 0 & 1 \\
\end{bmatrix}
\begin{bmatrix}
    A \\ B \\ C \\ D \\ E \\ F\\
\end{bmatrix}
=
\begin{bmatrix}
    1 \\ 1 \\ 1 \\ 1 \\ 1 \\
\end{bmatrix}
\]

Macierz główna nie jest macierzą kwadratową (a przynajmniej nie zawsze), 
więc algorytm nie może określić dokładnych wartości wszystkich zmiennych. Może jednak 
przekształcić tę macierz za pomocą operacji elementarnych do takiej postaci, żeby
mógł odczytać z niej jak najwięcej wartości zmiennych. \\
\[
\left[
\begin{array}{cccccc|c}
    1 & 1 & 0 & 0 & 0 & 0 & 1\\
    1 & 1 & 1 & 0 & 0 & 0 & 1\\
    0 & 0 & 0 & 1 & 1 & 0 & 1 \\
    0 & 0 & 0 & 0 & 1 & 1 & 1\\
    1 & 1 & 1 & 0 & 0 & 1 & 1\\
\end{array}
\right]
\longrightarrow
\left[
\begin{array}{cccccc|c}
    1 & 1 & 0 & 0 & 0 & 0 & 1\\
    0 & 0 & 1 & 0 & 0 & 0 & 0\\
    0 & 0 & 0 & 1 & 0 & 0 & 0 \\
    0 & 0 & 0 & 0 & 1 & 0 & 1\\
    0 & 0 & 0 & 0 & 0 & 1 & 0\\
\end{array}
\right]
\] \\
Co jest tożsame z układem równań:\\
$$
\begin{cases}
 A + B = 1 \\
 C = 0 \\
 D = 0 \\
 E = 1 \\
 F = 0 \\
\end{cases}
$$
\hfill \break
W tym przypadku doskonale widać, że pole E zawiera minę, natomiast pola C, D, F jej nie zawierają. 
O polach A i B nic nie wiadomo, co wynika z faktu iż macierz główna nie jest kwadratowa. \\
Warto jednak zauważyć, że operacje podstawowe nie zawsze zwrócą wartość 0 lub 1. W ogólności
algorytm sprawdza otrzymane wyniki za pomocą prostych warunków: \\
$$
\begin{cases}
 \text{if } \forall{(a_{i} \in M_{j}^W)} \text{ } |\sum{a_{i}}| = \sum{|a_{i}| = |b_{j}|}  \text{ then } x_{j}=1 \\
 \text{if } \forall{(a_{i} \in M_{j}^W)} \text{ } |\sum{a_{i}}| = \sum{|a_{i}|} = 0 \text{ then } x_{j}=0 \\
\end{cases}
$$
\hfill \break
Analogicznie jak wyżej, wszystkie pola z wartością 1 algorytm oznacza jako pola "z miną", a te z wartością 0 jako "bez miny".

\subsubsection*{Strzelanie w losowe pole}
Ta metoda jest używana wtedy i tylko wtedy, gdy obie powyższe po przejrzeniu całej planszy
nie dadzą rady oznaczyć ani jednego pola. Algorytm wybiera losowo jedno z nieoznaczonych pól
i oznacza je jako pole "bez miny". Jeśli oznaczy poprawnie gra jest kontynuowana z wykorzystaniem
 powyższych metod. Jeśli pole nie zostało poprawnie oznaczone, gra jest zakończona porażką.

\subsection{Sieć neuronowa}
\subsubsection*{Dane uczące}
Dane uczące wygenerowane zostały przy pomocy bota. Po wykonaniu obliczeń oznacza on miny w sposób poprawny, błędy występują w przypadku, gdy obecny stan planszy
nie pozwala na wyliczenie ich położenia. Pomyłki występują wyłącznie w przypadku oddania strzału, dane po niecelnym strzale nie zostawały uwzględnione. Po wybraniu pola na planszy, 
tworzono do 16 macierzy rozmiaru 4x4, które uwzględniały położenie pola na każdej z 16 możliwych pozycji. Miało to na celu nauczenie sieci odpowiednich wzorców z uwzględnieniem otoczenia wybranego
pola. Rozmiar macierzy 5x5 mogłby być zbyt duży w przypadku małej planszy (10x10), a wymiar 3x3 mógłby nie dostarczać odpowiedniej ilości informacji. Macierze przekształcano na wektory o długości
16, które stanowiły dane do trenowania sieci. Sieć wykorzystano do klasyfikacji. Etykiety stanowiło położenie miny w wektorze (0-15) lub liczba 16 w przypadku braku miny na przekazanym fragmencie planszy.
Wygenerowano około 1.000.000 wektorów treningowych, w tym po około 500.000 przypadków danych pozytywnych i negatywnych.\\
\begin{figure}[H]
    \centering
    \includegraphics[scale=0.75]{generating.png}
    \caption{Sposób generowania danych wejściowych.}
\end{figure}

\subsubsection*{Opis sieci}
Sieć składa się z czterech warstw gęstych o 512 neuronach i warstwy gęstej wyjściowej o 17 neuronach. Zastosowanie warstwy zapobiegającej przetrenowaniu nieznacznie pogarsza skuteczność sieci przy zastosowaniu danych testowych (ilość 50.000), 
więc finalnie nie została ona użyta. 
\begin{figure}[H]
    \centering
    \includegraphics[scale=0.85]{gaussian.png}
    \caption{Wykresy skuteczności danych testowych z zastosowaniem warstwy zapobiegajacej przetrenowaniu i bez jej użycia.}
\end{figure} 
Jako funkcji straty użyto funkcji sparse categorical crossentropy, jako funkcji optymalizującej funkcji adam. Wykresy skuteczności i wartości funkcji straty dla kolejnych epok podczas treningu:
\begin{figure}[H]
    \centering
    \includegraphics[scale=0.6]{skutecznosc.png}
    \caption{Wykres skuteczności w czasie treningu w kolejnych epokach.}
\end{figure} 
\begin{figure}[H]
    \centering
    \includegraphics[scale=0.6]{funkcjastraty.png}
    \caption{Wykres wartości funkcji straty podczas treningu w kolejnych epokach.}
\end{figure} 
\subsubsection*{Rozwiązywanie}
Dla każdego odkrytego pola z wartością numeryczną generowane jest do 16 macierzy 4x4. Korzystając z prawdopodobieństw przynależności do każdej z klas (oprócz ostatniej) zwróconych przez sieć neuronową, uzupełniana jest tablica wielkości planszy,
w której zawarte są prawdopodobieństwa wystąpenia miny na danym polu. Po oznaczeniu pola, na którym z największym prawdopodobieństwem znajduje się mina algorytm sprawdza, czy występują jakiekolwiek pola, dla których wszystkie miny zostały już znalezione. W przypadku sukcesu, 
odsłaniane są pozostałe pola z nimi graniczące. W przypadku wykrycia pola, z którym graniczy więcej min niż wskazuje jego wartość numeryczna, wszystkie graniczne miny są odznaczane z planszy. 
\subsection{Prolog}
Kolejna metoda oparta jest o proste reguły logiczne napisane w języku logicznym bazującym na prologu (pyDataLog).
Metoda iteruje po wszystkich polach sąsiadujących z miejscami, gdzie potencjalnie mogłaby znajdować się mina.
Dla każdego pola zlicza ona ilość sąsiadujących nieodkrytych pól oraz tych oznaczonych flagą, i na tej podstawie wnioskuje czy w sąsiadujących nieodsłoniętych polach znajduje się mina czy nie.
Reguły są opisane w następujący sposób:
\begin{itemize}
    \item{Gdy liczba pól oznaczonych flagą równa się wartości sprawdzonego pola, to wszystkie inne puste pola zostają odkryte.}
    \item{Gdy liczba pól nieodkrytyh zsumowana z liczbą pól oznaczonych flagą daje w wyniku wartość sprawdzonego pola, to na sąsiadujących pustych polach są miny, więc zostają one oznaczone flagą.}
\end{itemize}
Te dwie zasady mogą rozwiązać większość sytuacji, które występują na planszy, co daje efektywność rozwiązywania 65\% do 70\% wygranych gier na poziomie podstawowym.
Metoda jednak nie radzi sobie z trudniejszymi poziomami, które w znacznej większości gier wymagają umiejętności myślenia w przód, czego podane proste reguły logiczne nie potrafią
\hfill \break
\section{Porównanie metod}
\begin{center} 
    \begin{tabular}  { | l | c | r |   }
        \hline
        Algorytm & Mała & Duża  \\
        \hline
        Bot & 85-90 & 10-15 \\
        \hline
        Sieć neuronowa & 90-95 & 0 \\
        \hline
        Logika / prolog & 60-70 & 0  \\
        \hline
    \end{tabular}
\end{center}
\section{Wnioski}
W przypadku dużej planszy metoda rozwiązywania z pomocą sieci neuronowej jest nieskuteczna. Bot oznacza miny na tej planszy poprawnie, 
myli się w przypadku, gdy stan planszy nie pozwala na wyliczenie ich położenia i jest zmuszony do oddania strzału. Skuteczność bota mogłaby zostać poprawiona, gdyby
brać pod uwagę prawdopodobieństwo obecności miny na danym polu, występują jednak sytuacje, w których obecność miny na jednym z danych pól jest równoprawdopodobna.\\
  Skuteczność wytrenowanej sieci
wynosi około 75\%, więc nie jest ona w stanie oznaczyć wszystkich min tak dokładnie jak bot, co wpływa na końcową skuteczność metody.
W tym przypadku, bardziej skuteczne mogłoby być wykorzystanie sieci neuronowej z regresją, która sama uczyłaby się zasad gry, albo algorytmu 
genetycznego, który starałby się nie doprowadzić do sytuacji, w której musi być oddany strzał. Sieć neuronowa z klasyfikacją może nie być w stanie rozwiązać problemu.\\
Wyniki przedstawione w tabeli powyżej mogą byc przekłamane dla dużej planszy, ze względu na niewystarczającą ilość
przeprowadzonych testów. Wykonana w dany sposób obsługa okna przeglądarki jest kosztowna czasowo, co wpływa na czas wykonywania testów.


\end{document}
